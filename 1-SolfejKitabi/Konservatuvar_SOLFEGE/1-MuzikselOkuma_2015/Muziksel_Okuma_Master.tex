\documentclass[a4paper,12pt]{book}
\usepackage[turkish]{babel}
\usepackage[T1]{fontenc}
\usepackage[utf8]{inputenc}

\usepackage{lmodern}
\usepackage{anysize}
\usepackage{indentfirst}

\title{\textbf{{\huge Müziksel Okuma Çalışmaları}}  \\ \textit{{\Large Öğrenci Çalışma Kitabı}} }
\author{\textit{Yazan:} \textbf{Prof. Server ACİM}}
\date{2016 - MALATYA}

\begin{document}

\maketitle
\tableofcontents


\chapter{Giriş}
Ülkemizde \textbf{Müzik} adına eğitim veren kurumlarda \textit{(Müzik Öğretmenliği Programları, Güzel Sanatlar Fakültesi - Müzik Bölümleri, Sanat ve Tasarım Fakültesi veya Müzik ve Sahne Sanatları Fakültesi gibi adlarla bilinen kalıp başlıkların dışındaki birimlerde...}) eğer öğrencinin \textbf{Güzel Sanatlar ve Spor Lisesi} geçmişi yok ise, Konservatuvar'larda adına "Solfej" ve/veya "Solfej-Dikte-Teori" denilen; Müzik Öğretmenliği Bilim Dalı kökenli kişilerin "Müziksel İşitme-Okuma-Yazma" olarak adlandırdıkları derslerde kullanılan okuma parçaları iki örneğin dışında Alber LAVINGAC'ın 8-13 yaş arasında müzik öğrenimi gören kişiler için yazmış olduğu parçalardır. Bu iki örnek Muammer SUN öğretmenimizin "Solfej - 1" ve "Solfej - 2" parçaları ile Ali SEVGİ öğretmenimizin Makam temelli Solfej okuma parçalarıdır. Bunun dışında Pozzoli'nin piyano eşlikli okuma parçaları, "Bona" denilen okuma parçaları ise ayrı birer seçenektir.

Ben bir \textbf{besteci} olarak, bugüne kadar \textit{Solfej Eğitimi} adına üretilmiş olan tüm kitaplara üreten kişilere saygılarımı sunuyorum ve ben kendime göre ürettiğim yeni bir alternatifi kamuoyunun kullanımına sunuyorum bu çalışma ile. Dikkat ederseniz "çalışma" terimini kullanıyorum. \textbf{Müziksel Okuma Çalışmaları} adını taşıyan bu kitapta \textit{"müzik eğitimi"} alan tüm \textit{Müzik Öğretmenliği} öğrencileri, \textit{Güzel Sanatlar Fakültesi - Müzik Bölümü} öğrencilerine, \textit{Konservatuvar} öğrencilerinin faydalanmaları için ürettiğim ezgisel ve tartımsal okuma parçalarını kamuoyunun kullanımına sunuyorum. Ayrıca, bu kitabı \textit{Anadolu Güzel Sanatlar ve Spor Lisesi Müzik Öğrencileri ve Müzik Öğretmenleri} de kullanmak isterler ise elbette buna dair bir önlem yoktur.

Bu çalışma kitabına İngilizce olarak \textit{"Student's Workbook"} diyebiliriz. Bu kitabı öğrencilerine çalıştıran öğretmenler için de ayrı bir kitap üretmiş bulunuyorum. Bu kitaba da İngilizce olarak \textit{"Teacher's Book"} diyebiliriz. Öğretmenlerin el kitabında ayrıca Dikte parçaları yer almıştır. Bu ikinci kitapta ayrıca \textit{"Hazırlayıcı Dikte"} çalışmaları da yer almaktadır. Diktenin öğrencilere öğretilmesi, çalıştırılması konusunda öneriler de yer almaktadır bu ikinci kitapta.







\end{document}
