\documentclass[a4paper,12pt]{book}
\usepackage[turkish]{babel}
\usepackage[T1]{fontenc}
\usepackage[utf8]{inputenc}

\usepackage{lmodern}
\usepackage{anysize}

\title{Müziksel İşitme Çalışmaları}
\author{Prof. Server ACİM}

\begin{document}

\maketitle
\tableofcontents

%\begin{abstract}
%Burası özet bölgesidir.
%\end{abstract}

\chapter{Giriş}
Ülkemizde \textbf{Müzik} adına eğitim veren kurumlarda \texttt{(Müzik Öğretmenliği Programları, Güzel Sanatlar Fakültesi - Müzik Bölümleri, Sanat ve Tasarım Fakültesi veya Müzik ve Sahne Sanatları Fakültesi gibi adlarla bilinen kalıp başlıkların dışındaki birimlerde...}) eğer öğrencinin \textbf{Güzel Sanatlar ve Spor Lisesi} geçmişi yok ise, Konservatuvar'larda adına "Solfej" ve/veya "Solfej-Dikte-Teori" denilen; Müzik Öğretmenliği Bilim Dalı kökenli kişilerin "Müziksel İşitme-Okuma-Yazma" olarak adlandırdıkları derslerde kullanılan okuma parçaları iki örneğin dışında Alber LAVINGAC'ın 8-13 yaş arasında müzik öğrenimi gören kişiler için yazmış olduğu parçalardır. Bu iki örnek Muammer SUN öğretmenimizin "Solfej - 1" ve "Solfej - 2" parçaları ile Ali SEVGİ öğretmenimizin Makam temelli Solfej okuma parçalarıdır. 



\section{Notalar 1}
Bu Solfej kitabının hazırlanışını tek başıma yine LaTeX ve GNU/LilyPond ile çalışmak istiyorum. Kitabın yine Gece Kitaplığı üzerinden çıkması gerektiğine inanıyorum. Yalnız kitabı iki ayrı yapıda yayınlamak istiyorum. Şöyle ki:

\begin{itemize}
  \item Öğrenci Okuma Kitabı
  \item Öğretmen Kılavuz ve Dikte Kitabı
\end{itemize}

Çünkü bu kitap hem öğrenciler için malzemeler içerecek. Bu malzemeler genellikle okuma parçalarının notaları olacak. Dikte için öğrencilere tavsiyeler ve öğretmenlere tavsiyeler olacak. \textit{Öğretmen Kılavuz ve Dikte Kitabı}'nda ise öğretmene yönelik bir planlama içeren kaynak olacak ve ayrıca dikte notaları burada olacak. \textbf{Bu bir denemedir.}




\end{document}
