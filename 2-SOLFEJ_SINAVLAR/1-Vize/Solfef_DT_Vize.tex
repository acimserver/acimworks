\documentclass[a4paper,addpoints,12pt]{exam}
\usepackage[turkish]{babel}
\usepackage[utf8]{inputenc}
\usepackage{float}
\usepackage{graphicx}
\graphicspath{{images/}}
\usepackage{epstopdf}
\usepackage{amssymb}
\pointpoints{Puan}{Puan}
\qformat{\textbf{Soru \thequestion}\quad (\thepoints)\hfill}

\pagestyle{headandfoot}
\pagestyle{headandfoot}
\firstpageheader{\textbf{Solfej-Dikte-Teori-1}}{\textbf{Teori Sınavı}}{\textbf{16.11.2015}}
\runningheader{}{}{}
\firstpagefooter{}{}{}
\runningfooter{Solfej-Dikte-Teori-1}{Teori Sınavı-16.11.2015}{Sayfa \thepage\ / Toplam: \numpages}
\runningfootrule

\begin{document}
\begin{center}
\fbox{\fbox{\parbox{5.5in}{\centering
İnönü Üniversitesi Devlet Konservatuvarı - Müzik Bölümü\\
Solfej-Dikte-Teori-1 Dersi \\ Teori Sınavı - \textbf{Vize}
}}}
\end{center}
\vspace{0.1in}
\makebox[\textwidth]{Adınız ve Soyadınız:\enspace\hrulefill}
\vspace{0.1in}
\makebox[\textwidth]{İmza:\enspace\hrulefill}
\vspace{0.1in}
%%%%%%%%%%%%%%%%%%%%%%%%%%%%%%%%%%%%%%%%%%%%%%%%%%%%%%%%%%%%%%%%%%%%%%%%%%%%%%%%%%%5
%\begin{center}
%\underline{\textbf{Türkçe'den İngilizce'ye çeviri soruları}}  
%\end{center}

\begin{questions}
%Soru 1-5:
\question[50]
\textbf{Aşağıda notaları verilmiş olan aralıkların niteliklerini yazınız.}

%Seçeneklerin alt satırda gözükmesi için, buraya 1 satır boşluk bırakmalısın.
%\makeemptybox{6.6in}
%\break
%Soru 6-10:
\question[50]
\textbf{Aşağıda ilk sesi ve NİTELİĞİ verilmiş olan aralıkların oluşması için gereken ikinci sesleri yazınız.}

%Seçeneklerin alt satırda gözükmesi için, buraya 1 satır boşluk bırakmalısın.


%Soru 11-15:
\question[50]
\textbf{Aşağıda donanımı verilmiş olan tonlar için gerekli cevabı ilgili yere yazınız.}


%Soru 16-20:
\question[50]
\textbf{Aşağıda tonalitesi belirtilmiş olan majör ve/veya minör tonlar için gereken donanım işaretlerini gerekli yerlere yazınız.}

%Soru 21-25
\question[50]
\textbf{Aşağıda notası verilmiş olan yarım perdelerin "niteliklerini" ilgili yere yazınız.}

%Soru 26-30
\question[50]
\textbf{Aşağıda niteliği belirtilmiş olan yarım perdenin oluşması için gereken ikinci sesleri ilgili yerlere yazınız.}






%\makeemptybox{7.9in}

\end{questions}

\end{document}
